\documentclass[12pt]{extarticle}
\documentclass{article}
\usepackage{amssymb}
\usepackage{array}
\usepackage{booktabs}
\usepackage{enumitem}
\usepackage{fancyhdr}
\usepackage[utf8]{inputenc}
\usepackage{geometry}
\usepackage{graphicx}
\usepackage{xcolor}
\usepackage[table]{xcolor}
\usepackage[most]{tcolorbox}
\geometry{
    left=1cm,
    right=1cm,
    top=1cm,
    bottom=1cm
}
\fancypagestyle{plain}{
    \fancyhf{}
    \renewcommand{\headrulewidth}{0pt}
    \fancyfoot[C]{\thepage}
}
\pagestyle{fancy}
\fancyhf{}
\renewcommand{\headrulewidth}{0pt}
\fancyfoot[C]{\thepage}
\centering
\fancyhead[R]{Ce document est une propriété privée et confidentielle. Toute diffusion non autorisée est interdite.}
\flushleft
\begin{document}

% ====================
% PAGE DE GARDE
% ====================
\begin{titlepage}
\centering
\vspace*{4cm}

{\LARGE \textbf{Rapport d’audit de sécurité}}\\[0.5cm]
{\Large Conforme au référentiel PASSI (ANSSI)}\\[2cm]

\textbf{Mission :} {{ mission.mission }}\\
\textbf{Client :} {{ mission.client_company }}\\
\textbf{Date :} \today\\[2cm]

\vfill
\textit{Ce rapport est confidentiel et destiné exclusivement au commanditaire.}
\end{titlepage}

\newpage

% ====================
% TABLE DES MATIÈRES
% ====================
\tableofcontents
\newpage

% ====================
% 1. CONTEXTE
% ====================
\section{Contexte et objectifs}
\label{sec:contexte}

Ce rapport présente les résultats de la mission d’audit de sécurité réalisée
dans le cadre du référentiel PASSI défini par l’ANSSI.

Les objectifs de la mission sont :
\begin{itemize}
\item Évaluer le niveau de sécurité du périmètre audité
\item Identifier les écarts vis-à-vis du référentiel
\item Formuler des recommandations adaptées
\end{itemize}

% ====================
% 2. PÉRIMÈTRE
% ====================
\section{Périmètre de la mission}
\label{sec:perimetre}

Le périmètre de l’audit couvre les domaines et chapitres décrits ci-après.

% ====================
% 3. MÉTHODOLOGIE
% ====================
\section{Méthodologie et référentiel}
\label{sec:methodo}

La mission a été conduite conformément aux exigences du PASSI :
\begin{itemize}
\item Analyse documentaire
\item Entretiens
\item Vérifications techniques
\item Évaluation qualitative et quantitative
\end{itemize}

% ====================
% 4. SYNTHÈSE DES ÉVALUATIONS
% ====================
\section{Synthèse des évaluations}
\label{sec:synthese}

\begin{longtable}{p{6cm} c c}
\toprule
\textbf{Chapitre} & \textbf{Score} & \textbf{Lien} \\
\midrule


{{ chapitre.chapitre }} &
{{ chapitre.score_moyen }} &
\hyperref[chapitre:{{ chapitre.id_chapitre }}]{Voir} \\


\bottomrule
\end{longtable}

% ====================
% 5. RÉSULTATS DÉTAILLÉS
% ====================
\section{Résultats détaillés}
\label{sec:resultats}


\subsection{Domaine : {{ domaine.domaine }}}
\label{domaine:{{ domaine.id_domaine }}}


\subsubsection{Chapitre : {{ chapitre.chapitre }}}
\label{chapitre:{{ chapitre.id_chapitre }}}



\label{eval:{{ chapitre.id_chapitre }}:{{ loop.index }}}
{Question :} {{ q.question }}
% Peut être rajouter une description plus tard qui explique le contexte de la question ? 
{{ q.reponse }}
{{ q.objectif }}


\paragraph{Éléments de preuve}
\includegraphics[scale=0.5]{ {{- q.piece_jointe -}} }


\paragraph{Évaluation}
\textbf{Score :} {{ q.evaluation }} — \score{{ q.evaluation }}




\medskip





% ====================
% 6. CONCLUSION
% ====================
\section{Conclusion}
\label{sec:conclusion}

L’audit met en évidence des axes d’amélioration prioritaires pour renforcer
le niveau de sécurité global du périmètre audité.

% ====================
% ANNEXES
% ====================
\appendix
\section{Glossaire}
PASSI : Prestataires d’Audit de la Sécurité des Systèmes d’Information.

\end{document}
